\section{Conclusion}
\label{sec:conclusion}

To conclude, the main underlying philosophy followed during the development of all the approaches tried was related in finding a valid solution through iterations, each of which was capable of improving the previously best one found. This process resulted in the discovery of a solution composed of a \textbf{hybrid recommendation system} leveraging a linear combination of \textbf{Expanded-Item-Item Collaborative Filtering} and \textbf{Compact Item-Item Collaborative Filtering} based on \textbf{query result cardinality}. There was also the intent of the development of a modular solution. The use of shared components and algorithms between the various described sub-tasks demonstrates this.

The proposed solution was able, at the expense of slower performance, to \textbf{slightly improve} in correctness all the other solutions with standard implementations or developed from scratch. The improvement was most noticeable during evaluations in which the \textit{Utility Matrix} filled by the proposed solution for both PART A and PART B was compared, using the \textbf{three evaluation metrics}, with the actual \textit{Real Complete Utility Matrix} generated during the dataset construction. In terms of the evaluation of the \textit{TOP\_K} queries of each user computed by the proposed solution, performed using the \textbf{Jaccard Similarity}, there was almost \textbf{no significant advantage} in using the hybrid solution over the baselines.

In conclusion, the project's development was useful in understanding that in order to create a successful recommendation system, it is often necessary to rely on multiple approaches to achieve a better final outcome. There is definitely \textbf{room for improvement}, both in terms of the analysis of smart combinations of components and larger-scale testing.








